\section{Conclusions}\label{sec:conclusions}
%setting and problem
Scientific imaging plays a major role in many clinical applications and biomedical studies.
However, image segmentation is a time-consuming process when done manually, and often experts are needed to identify structures correctly~\autocite{Litjens2017}.
But even when automatic segmentation algorithms are used, at least some manual segmentations are needed, since to date all state-of-the-art algorithms for scientific image segmentation are supervised and rely on annotated training data~\autocite{Litjens2017}.
The aim of this project thus was to adapt an unsupervised algorithm \gls{stego}~\autocite{Hamilton2022} to be used with high-resolution scientific imaging data and evaluate this method on a scientific imaging data set (depicting degradation of screws in bones).

% solution we found
It was shown that \gls{stego} can be used with scientific data, and specifically training a model on a data set significantly improved predictions in comparison to an arbitrary (pre-trained) model.
%lessons learned
%problems
However, the resulting predictions can not yet be used in a real-world science context.
Particularly one label (\formatLabel{degraded screw}) was not detected at all rate, which might hint at the fact that the features received from the backbone might not embed this label well.
Moreover, because the divergence between cross-validation folds was similar to divergence between training configurations, no clear suggestions regarding training configurations for scientific data could be derived.
%next steps
To mitigate these shortcomings, the next steps should include a grid search for better training parameters, and training the backbone (which was adjusted for photographs, not scientific images), to reach a better feature embedding.

%findings and potential
Given that \gls{stego} is an unsupervised algorithm, results are impressive, as explained earlier.
Additionally, it must be considered that unsupervised algorithms identify structures in the data, that do not necessarily align with the structures that were represented in the ground truth.
So aberration from the ground truth has to be expected to some degree.
And given that \gls{stego} was used for scientific segmentation, which it was not designed for, the results are very encouraging. % with a backbone that was also not designed for scientific images,
Achieving these promising results highlights the potential of \gls{stego} to be further adapted in the future for unsupervised semantic segmentation of scientific images.

% so footnotes are placed reasonably
\clearpage 
